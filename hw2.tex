\documentclass[english]{article}

\usepackage[unicode=true]{hyperref}

%%% Hyperref: links should appear blue as you would expect
\hypersetup{
  colorlinks=true,
  linkcolor=blue,
  urlcolor=blue,
  filecolor=blue,
}

%%% Listings: use to include code in your solutions
% See: https://www.overleaf.com/learn/latex/Code_listing
\usepackage{listings}

%%% Listings: setup defaults for code formatting
\lstset{
  language={C++},
  frame=tb,
  numbers=left,
  numberstyle=\tiny,
  basicstyle=\small\sffamily,
  breaklines=true,
}

%%% GraphicX: insert images (e.g. screenshots) into the document
% See: https://www.overleaf.com/learn/latex/Inserting_Images
\usepackage{graphicx}

%%% Float: Allows the use of the H specifier in images to force them in place
\usepackage{float}


%%% Enumitem: alphabetic enumerations with improved syntax
% E.g.
% (a) ...
% (b) ...
\usepackage[shortlabels]{enumitem}


%%% AMS Math: access to math environments alike align
% Aligning: https://www.overleaf.com/learn/latex/Aligning_equations_with_amsmath 
\usepackage{amsmath}

% Additional options when using tables
\usepackage{array}

%%% Custom commands

\newcounter{problemi}
\setcounter{problemi}{1}

\newcommand*{\headfont}{\fontsize{1.1em}{1.0em}\selectfont}


\newcommand\problem[1]{
  \noindent{\headfont\\\theproblemi. #1}\\
  \stepcounter{problemi}\smallskip
}

\newcommand*{\code}[1]{\texttt{#1}}


\begin{document}

%%%%%%%%%%%%%%%%%%%%%%%%%%%%%%% COVER PAGE %%%%%%%%%%%%%%%%%%%%%%%%%%%%%%%%%%%%%

\begin{centering}
    {\Large CSCE 221 Cover Page} \\ \medskip    
\end{centering}

Please list all sources in the table below including web pages which you used to solve or implement the current homework. If you fail to cite sources you can get a lower number of points or even zero, read more Aggie Honor System Office \url{https://aggiehonor.tamu.edu/} \\

% EDIT: the below information appropriately
\noindent
\begin{center}
    {\large
    \begin{tabular}{|p{0.35\linewidth}|p{0.45\linewidth}|} \hline
        Name          & Andrew Shang       \\ \hline
        UIN           & 830002330         \\ \hline
        Email address & ashang1524@tamu.edu      \\ \hline
    \end{tabular}
    }
\end{center}

Cite your sources using the table below. Interactions with TAs and resources presented in lecture do not have to be cited.
\noindent
\begin{center}
    {\large
    \begin{tabular}{|>{\centering\arraybackslash}m{0.25\linewidth}|m{0.70\textwidth}|} \hline
        {\large People}            &
            \begin{enumerate}
                % EDIT
                % People: Add sources below as items
                \item None
            \end{enumerate}
        \\ \hline
        {\large Webpages}          & 
            \begin{enumerate}
                % EDIT
                % Webpages: Add sources below as items
                \item None
            \end{enumerate}
        \\ \hline
        {\large Printed Materials} &
            \begin{enumerate}
                % EDIT
                % Printed Material: Add sources below as items
                \item None
            \end{enumerate}
        \\ \hline
        {\large Other Sources}     &
            \begin{enumerate}
                % EDIT
                % Other sources: Add sources below as items
                \item None
            \end{enumerate}
        \\ \hline
    \end{tabular}
    }
\end{center}

\pagebreak

%%%%%%%%%%%%%%%%%%%%%%%%%%%%%%% HOMEWORK TWO %%%%%%%%%%%%%%%%%%%%%%%%%%%%%%%%%%%%%

\begin{centering}
    {\Huge Homework 2}\\ \bigskip
    {\Large Due March 25 at 11:59 PM}\\ \bigskip
\end{centering}

\textbf{Typeset your solutions to the homework problems preferably in \LaTeX or LyX.
See the class webpage for information about their installation and tutorials.}

%%%%%%%%%%%%%%%%%%%%%%%%%%%%%%%%% PROBLEM 1 %%%%%%%%%%%%%%%%%%%%%%%%%%%%%%%%%%%%%%%

\problem{(15 points)
Provided two sorted lists, \code{l1} and \code{l2}, write a function in C++ to \emph{efficiently} compute \code{l1 $\cap$ l2} using only the basic STL list operations. The lists may be empty or contain a different number of elements e.g $|\code{l1}| \neq |\code{l2}|$. You may assume \code{l1} and \code{l2} will not contain duplicate elements.
}

\noindent Examples (all set members are list node):
\begin{itemize}
  \item $\{1, 2, 3, 4\} \cap \{2, 3\} = \{2, 3\}$
  \item $\emptyset \cap \{2, 3\} = \emptyset$
  \item $\{2, 9, 14\} \cap \{1, 7, 15\} = \emptyset$
\end{itemize}

\bigskip

\begin{enumerate}[(a)]
  \item Complete the function below. Do not use any routines from the algorithm header file.

\begin{lstlisting}
#include <list>

std::list<int> intersection(const std::list<int> & l1, const std::list<int> & l2) {

  list<int> both;
  auto astart = l1.begin();
  auto bstart = l2.begin();

  while (astart != l1.end() && bstart != l2.end()) {
    if (*astart < *bstart) {
      astart++;
    } else if (*bstart < *astart) {
      bstart++;
    } else {
      if (both.back() != *astart || *astart == 0) {
        both.push_back(*astart);
      }
      astart++;
      bstart++;
    }
  }

  return both;
}
\end{lstlisting}

  \item Verify that your implementation works properly by writing two test cases. Provide screenshot(s) with the results of your testing.

  \begin{figure}[H]
     \centering
     \includegraphics[width=0.5\linewidth]{q1.jpg}
     \caption{Results of testing the \code{intersection} function}%
     \label{fig:intersection_tests}
  \end{figure}  

  \item What is the running time of your algorithm? Provide a big-O bound. Justify.

  \begin{equation}
    f(n) \in O(min(l1, l2))
  \end{equation}
  
  % You may also want a derivation in your justification
  For any given intersection, such numbers must be included in both initial sets. Thus, if any given set were greater in length than the other, additional numbers would not be included. Thus is it only necessary to loop through the smaller set checking for any intersections.

\end{enumerate}

%%%%%%%%%%%%%%%%%%%%%%%%%%%%%%%%% PROBLEM 2 %%%%%%%%%%%%%%%%%%%%%%%%%%%%%%%%%%%%%%%

\problem{(15 points)
  Write a C++ recursive function that counts the number of nodes in a singly linked list. Do not modify the list.
}

Examples:
\begin{itemize}
  \item \code{count\_nodes($ (2) \rightarrow (4) \rightarrow (3) \rightarrow \text{nullptr}$)} = 3
  \item \code{count\_nodes($\text{nullptr}$)} = 0
\end{itemize}

\bigskip

\begin{enumerate}[(a)]
  \item Complete the function below:
  \label{enum:recursive_node_imp}

\begin{lstlisting}
template<typename T>
struct Node {
  Node * next;
  T obj;

  Node(T obj, Node * next = nullptr)
    : obj(obj), next(next)
  { }
};

template<typename T>
int count_nodes(T* node) {
  if (node != nullptr) {
    return 1 + count_nodes(node->next);
  }
}
\end{lstlisting}

  \item Verify that your implementation works properly by writing two test cases for the function you completed in part \ref{enum:recursive_node_imp}. Provide screenshot(s) with the results of your testing.

\begin{figure}[H]
  \centering
  \includegraphics[width=0.5\linewidth]{q2.jpg}
  \caption{Results of testing the \code{count\_nodes} function}%
  \label{fig:count_node_tests}
\end{figure}  

  \item Write a recurrence relation that represents your algorithm.
  
  \begin{equation}
    T(n) = \begin{cases}
      = 0, & \text{if } n = 1 \\
      = T(n - 1) + C, & \text{if } n = n
    \end{cases}
  \end{equation}

  \item Solve the recurrence relation using the iterating or recursive tree method to obtain the running time of the algorithm in Big-O notation.

  \begin{align}
    T(n) &= T(n - 1) + C \\
         &= T(n - 2) + 2C \\
         &= T(n - 3) + 3C \\
         & \cdots \\
         & T(n - n) + nC \\
         & O(n)
  \end{align}

\end{enumerate}

%%%%%%%%%%%%%%%%%%%%%%%%%%%%%%%%% PROBLEM 3 %%%%%%%%%%%%%%%%%%%%%%%%%%%%%%%%%%%%%%%

\problem{(15 points)
  Write a C++ recursive function that finds the maximum value in an array 
  (or vector) of integers \emph{without} using any loops. You may assume the
  array will always contain at least one integer. Do not modify the array.
}

\begin{enumerate}[(a)]
  \item Complete the function below:

\begin{lstlisting}
#include <vector>

int find_max_value(int array[], int n) {
    if (n == 1) { return array[0]; }
    return max(array[n - 1], find_max_value(array, n- 1));
}

\end{lstlisting}

  \item Verify that your implementation works properly by writing two test cases. Provide screenshot(s) with the results of the tests.

  \begin{figure}[H]
    \centering
    \includegraphics[width=0.5\linewidth]{q3.jpg}
    \caption{Results of testing the \code{find\_max\_value} function}%
    \label{fig:find_max_value}
  \end{figure}  

  \item Write a recurrence relation that represents your algorithm.

  \begin{equation}
    T(n) = \begin{cases}
      = 0, & \text{if } n = 1 \\
      = T(n - 1) + C, & \text{if } n = n
    \end{cases}
  \end{equation}

  \item Solve the recurrence relation and obtain the running time of the algorithm in Big-O notation. Show your process.

  \begin{align}
    T(n) &= T(n - 1) + C \\
         &= T(n - 2) + 2C \\
         &= T(n - 3) + 3C \\
         & \cdots \\
         & T(n - n) + nC \\
         & O(n)
  \end{align}

\end{enumerate}

%%%%%%%%%%%%%%%%%%%%%%%%%%%%%%%%% PROBLEM 4 %%%%%%%%%%%%%%%%%%%%%%%%%%%%%%%%%%%%%%%

\problem{(15 points) 
  What is the best, worst and average running time of quick sort algorithm?
}

\begin{enumerate}[(a)]
  % QUESTION: Do they need to solve them?
  \item Provide recurrence relations. For the average case, you may assume that quick sort partitions the input into two halves proportional to $c$ and $1 - c$ on each iteration.
  \label{enum:recurrence_relation}

  Best:
  \begin{equation}
    T(n) = \begin{cases}
      = 0, & \text{if } n = 0 \\
      = 2T(n / 2), & \text{if } n = n
    \end{cases}
  \end{equation}

  Average:
  \begin{equation}
    T(n) = \begin{cases}
      = 0, & \text{if } n = 0 \\
      = T(n - 1), & \text{if } n = n
    \end{cases}
  \end{equation}
  
  Worst:
  \begin{equation}
    T(n) = \begin{cases}
      = 0, & \text{if } n = 0 \\
      = T(n - 1), & \text{if } n = n
    \end{cases}
  \end{equation}

  \item Solve each recurrence relation you provided in part \ref{enum:recurrence_relation}

  % You may want to include an image using the figure environment and solve it on paper
  % You could also use the align environment to typeset math
  % \begin{figure}[H]
  %   \centering
  %   \includegraphics[width=0.5\linewidth]{image.jpg}
  %   \caption{Solution to recurrence relations}%
  %   \label{fig:recurrence_solution}
  % \end{figure}  

  \item Provide an arrangement of the input array which results in each case. Assume the first item is always chosen as the pivot for each iteration.

    \begin{table}[H]
      \centering
      \begin{tabular}{c c}
        Best    & Picking the middle element as the pivot \\
        Average & Having an arbitrary element as the pivot \\
        Worst   & Largest and smallest element as the pivot
      \end{tabular}
    \end{table}

\end{enumerate}

%%%%%%%%%%%%%%%%%%%%%%%%%%%%%%%%% PROBLEM 5 %%%%%%%%%%%%%%%%%%%%%%%%%%%%%%%%%%%%%%%

\problem{(15 points)
  Write a C++ function that counts the total number of nodes with two children in a 
  binary tree (do not count nodes with one or none child). You can use a STL container 
  if you need to use an additional data structure to solve this problem. 
}
\begin{figure}[H]
  \centering
  \includegraphics[width=0.3\linewidth]{./binary_tree_example.png}
  \caption{Calling \code{count\_filled\_nodes} on the root node F returns \code{3}}%
  \label{fig:binary_tree_example}
\end{figure}

\begin{enumerate}[(a)]
  \item Complete the function below. The function will be called with the root node (e.g. \code{count\_filled\_nodes(root)}). The tree may be empty. Do not modify the tree.

\begin{lstlisting}
#include <vector>

template<typename T>
struct Node {
  Node<T> *left, *right;
  T obj;

  Node(T obj, Node<T> * left = nullptr, Node<T> * right = nullptr)
    : obj(obj), left(left), right(right)
  { }
};

template<typename T>
int count_filled_nodes(const Node<T> * node) {
  if (node && node->left && node->right) {
      return 1 + count_filled_nodes(node->left) + count_filled_nodes(node->right);
  }
  return 0;
}

\end{lstlisting}

  \item Use big-O notation to classify your algorithm. Show how you arrived at your answer.
  
  The big-O notation is $O(n)$ since it is simply checking all possible nodes in any given tree for nodes with 2 children. There is only 1 operation being done at each call of the recursive function, which means for $n$ nodes in a given tree, there are only $n$ operations completed.

  \begin{equation}
    f(n) \in O(n)
  \end{equation}

\end{enumerate}

%%%%%%%%%%%%%%%%%%%%%%%%%%%%%%%%% PROBLEM 6 %%%%%%%%%%%%%%%%%%%%%%%%%%%%%%%%%%%%%%%

\problem{(15 points)
  For the following statements about red-black trees, provide a justification for each 
  true statement and a counterexample for each false one.
}

\begin{enumerate}[(a)]
  \item A subtree of a red-black tree is itself a red-black tree.
  
  This is false because at any given red node, there cannot be a consecutive red node. Red nodes cannot have children, so at the worst case the number of nodes alternate in red and black order.

  \item The sibling of an external node is either external or red.
  
  This is true. An external node is such node that's null. This external node has to be black to begin with, as all external nodes are black. However, when calculating depth, all external nodes must have the same depth. If there were a black node that follows the original external node, this would break the depth rule. Thus, it can be possible for the sibling of an external node to be red as well.


  \item There is a unique 2-4 tree associated with a given red-black tree.
  
  This is true. There can be 2, 3, or 4 nodes in a 2-4 tree. For 2 nodes, we can have 2 pointers with 1 data element, which in that case is simply a node with no children. Likewise for 3 and 4 nodes, a red black tree with 1 and 2 red children respectively, satisfies this.


  \item There is a unique red-black tree associated with a given 2-4 tree.
  
  This is false. While this applies for even numbered nodes, for a 3 node, this can be implemented in 2 seperate ways: black, red, black and red, black, red.

\end{enumerate}

%%%%%%%%%%%%%%%%%%%%%%%%%%%%%%%%% PROBLEM 7 %%%%%%%%%%%%%%%%%%%%%%%%%%%%%%%%%%%%%%%

\problem{(10 points)
  Modify this skip list after performing the following series of operations: \code{erase(38)}, 
  \code{insert(48,x)}, \code{insert(24, y)}, \code{erase(42)}. Provided the recorded coin flips 
  for \code{x} and \code{y}. Provide a record of your work for partial credit.
}

\begin{tabular}{ccccccccccccc}
$-\infty$ &  & ----- &  & ----- &  & ----- &  & ----- &  & ----- &  & $+\infty$\tabularnewline
          &  &       &  &       &  &       &  &       &  &       &  & \tabularnewline
$-\infty$ &  & ----- &  & 17    &  & ----- &  & ----- &  & ----- &  & $+\infty$\tabularnewline
          &  &       &  &       &  &       &  &       &  &       &  & \tabularnewline
$-\infty$ &  & ----- &  & 17    &  & ----- &  & ----- &  & 42    &  & $+\infty$\tabularnewline
          &  &       &  &       &  &       &  &       &  &       &  & \tabularnewline
$-\infty$ &  & ----- &  & 17    &  & ----- &  & ----- &  & 42    &  & $+\infty$\tabularnewline
          &  &       &  &       &  &       &  &       &  &       &  & \tabularnewline
$-\infty$ &  & 12    &  & 17    &  & ----- &  & 38    &  & 42    &  & $+\infty$\tabularnewline
          &  &       &  &       &  &       &  &       &  &       &  & \tabularnewline
$-\infty$ &  & 12    &  & 17    &  & 20    &  & 38    &  & 42    &  & $+\infty$\tabularnewline
\end{tabular} \bigskip

\begin{figure}[H]
  \centering
  \includegraphics[width=0.5\linewidth]{q7.jpg}
\end{figure}

\end{document}
